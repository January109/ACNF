\documentclass[11pt]{report}
\usepackage{styles} 
\title{The analytic class number formula and the distribution of ideals in a number field}
\begin{document}
\maketitle
\abstract
Outline ACNF, show how distribution of ideals relates
\np
\tableofcontents
\chapter{Background} % Set up notation
Let $K / \Q$ be a finite extension, which we will refer to as a \emph{number field}. To each such field we have a \emph{ring of integers} $\o_K$ which plays a similar role to $\Z \subseteq \Q$, defined as the set (subring) of $x \in K$ which are roots of monic $f \in \Z[x]$. 

% Considerations: Either gloss past with verbal explanations, or give structured definition by definition

Definitions: Number field [embeddings] $K$, ring of integers $\o_K$, field norm $N$ [in terms of embeddings]

Fractional ideals, invertible ideals, class group $\Cl(K)$, class number $h$

Unit group $\o_K^*$, Dirichlet's unit theorem (roots of unity denoted $\mu_K$ of cardinality $\omega_K$), $\Log$, trace zero hyperplane and regulator (volume induced by taking measure corresponding to a coordinate projection); volume

The Dedekind zeta function (state both formulations)

State the theorem

\section{Geometry}
Geometric interpretation $K \hookrightarrow K_\R$ (and volume), discriminant, lattices, covolume of $\o_K$ and ideals

\chapter{The analytic class number formula}
\section{Series and continuations} % COME UP WITH BETTER NAMES
Lemma 1: $(a_i)$ with $\sum a_i = O(t^\sigma)$

Lemma 1.5: Riemann zeta admits a meromorphic continuation to $\Re(s) > 0$, simple pole at $s = 1$ of residue 1.

Lemma 2: $(a_i)$ with $\sum a_i = \rho t + O(t^\sigma)$, $\sigma \in [0, 1)$.

Motivation for why this is relevant
\section{The distribution of ideals}
Motivate by arguing for sufficiently nice boundary
\subsection{Lipschitz parametrisability}
Idea: count points in image of $\o_K$ under $\Log$, but need nice boundary

Lemma + corollary on computing with $(n - 1)$-Lipschitz parametrisable boundary.

\subsection{Integral ideals of bounded norm}
The long computation

\end{document}